\documentclass[letterpaper,11pt]{article}

%----------PACKAGES----------
\usepackage{latexsym}
\usepackage[empty]{fullpage}
\usepackage{titlesec}
\usepackage{marvosym}
\usepackage[usenames,dvipsnames]{color}
\usepackage{verbatim}
\usepackage{enumitem}
\usepackage[hidelinks]{hyperref}
\usepackage{fancyhdr}
\usepackage[english]{babel}
\usepackage{tabularx}
\usepackage{stix}
\usepackage{changepage}
\input{glyphtounicode}


%----------FONT OPTIONS----------
% Sans-serif options:
% \usepackage[sfdefault]{FiraSans}
% \usepackage[sfdefault]{roboto}
% \usepackage[sfdefault]{noto-sans}
% \usepackage[default]{sourcesanspro}
% 
% Serif options:
% \usepackage{CormorantGaramond}
% \usepackage{charter}

%----------PAGE SETUP----------
\pagestyle{fancy}
\fancyhf{} % clear all header and footer fields
\fancyfoot{}
\renewcommand{\headrulewidth}{0pt}
\renewcommand{\footrulewidth}{0pt}

% Adjust margins
\addtolength{\oddsidemargin}{-0.5in}
\addtolength{\evensidemargin}{-0.5in}
\addtolength{\textwidth}{1in}
\addtolength{\topmargin}{-.5in}
\addtolength{\textheight}{1.0in}

\urlstyle{same}
\raggedbottom
\raggedright
\setlength{\tabcolsep}{0in}

% Sections formatting
\titleformat{\section}{
  \vspace{-4pt}\scshape\raggedright\large
}{}{0em}{}[\color{black}\titlerule \vspace{-5pt}]

% Ensure that generated PDF is machine readable/ATS parsable
\pdfgentounicode=1

%----------CUSTOM COMMANDS----------
\newcommand{\resumeItem}[1]{
  \item\small{
    {#1 \vspace{-2pt}}
  }
}

\newcommand{\resumeSubheading}[4]{
  \vspace{-2pt}\item
    \begin{tabular*}{0.97\textwidth}[t]{l@{\extracolsep{\fill}}r}
      \textbf{#1} & #2 \\
      \textit{\small#3} & \textit{\small #4} \\
    \end{tabular*}\vspace{-7pt}
}

\newcommand{\resumeSubSubheading}[2]{
    \item
    \begin{tabular*}{0.97\textwidth}{l@{\extracolsep{\fill}}r}
      \textit{\small#1} & \textit{\small #2} \\
    \end{tabular*}\vspace{-7pt}
}

\newcommand{\resumeProjectHeading}[2]{
    \item
    \begin{tabular*}{0.97\textwidth}{l@{\extracolsep{\fill}}r}
      \small#1 & #2 \\
    \end{tabular*}\vspace{-7pt}
}

\newcommand{\resumeSubItem}[1]{\resumeItem{#1}\vspace{-4pt}}

\renewcommand\labelitemii{$\vcenter{\hbox{\tiny$\bullet$}}$}

\newcommand{\resumeSubHeadingListStart}{\begin{itemize}[leftmargin=0.15in, label={}]}
\newcommand{\resumeSubHeadingListEnd}{\end{itemize}}
\newcommand{\resumeItemListStart}{\begin{itemize}}
\newcommand{\resumeItemListEnd}{\end{itemize}\vspace{-5pt}}

\newcommand{\hex}{\begin{tiny} $\hexagonblack$ \end{tiny}}

%----------DOCUMENT BEGINS HERE----------
\begin{document}

\begin{center}
    \textbf{\Huge \scshape John I. Capone, Ph.D.} \\ \vspace{1pt}
    \small jicapone.optics@gmail.com $|$ 609-915-6213\\ \vspace{2pt}
\end{center}

\begin{adjustwidth}{50pt}{50pt}
\raggedright
\small Optical scientist with 14+ years designing systems including 10mm-scale freeform optics\\
and volume phase holographic gratings (VPHGs). Developer of GPU-accelerated Python tools for optical modeling. Extensive prototyping experience with manufacturing partners.
\end{adjustwidth}

%-----------SKILLS-----------
\section{Skills}
\begin{itemize}[leftmargin=0.15in, label={}]
    \small{
        \item\begin{minipage}[t]{\linewidth}
            \textbf{Optical Design \& Analysis}: \hangindent=50pt \hangafter=1 Compact systems, Holographic optics, Diffractive optics, Illumination, System architecture, End-to-end modeling, Tolerancing, Radiometry, Wavefront propagation, Interferometry, Stray light
        \end{minipage}
        \item\begin{minipage}[t]{\linewidth}
            \textbf{Programming \& Tools}: \hangindent=50pt \hangafter=1 Python, PyTorch, Custom ray tracing, Zemax OpticStudio (ZOS) API, RCWA/GSolver, ENZ, Git, FDTD, Code V, FRED, MATLAB, LabVIEW, SolidWorks, Autodesk Inventor
        \end{minipage}
        \item\begin{minipage}[t]{\linewidth}
            \textbf{Laboratory \& Prototyping}: \hangindent=50pt \hangafter=1 Prototype development, Manufacturing partnerships, Cleanroom operations, Optical AIT, Fizeau/LUPI interferometry, CMM metrology, Spectroradiometry, Rapid iteration, Cryogenic \& vacuum testing
        \end{minipage}
    }
\end{itemize}

%-----------EXPERIENCE-----------
\section{Experience}
\resumeSubHeadingListStart
    \resumeSubheading
      {Optical Design and Analysis Group Lead}{November 2024 -- Present}
      {NASA Goddard Space Flight Center}{Greenbelt, MD, USA}
      \resumeItemListStart
        \resumeItem{Support project staffing and technical oversight}
        \resumeItem{Organize and deliver technical talks to facilitate knowledge-sharing and promote cross-project learning}
        \resumeItem{Coordinate peer reviews to promote engineering rigor and collaboration}
      \resumeItemListEnd

    \resumeSubheading
      {Optics Lead for NASA payloads}{April 2021 -- Present}
      {NASA Goddard Space Flight Center}{Greenbelt, MD, USA}
      \resumeItemListStart
        \resumeItem{Create custom software tools, e.g., GPU accelerated ray tracing, ENZ analysis, heterodyne DWS signal simulation}
        \resumeItem{Exploring application of FDTD tool to UV diffraction problem}
        \resumeItem{Develop system architectures, interfaces, and requirements within cross-functional teams}
        \resumeItem{End-to-end design, modeling, and analysis of compact optical systems utilizing refractive elements and freeform surfaces}
        \resumeItem{Translate system requirements into manufacturable optical specifications}
        \resumeItem{Investigated 3D printing methods for optical surface prototyping, achieving 3x improvement in surface form}
        \resumeItem{Designed illumination systems including LED baffle optimization (CCRS), solar simulator OGSE (OSAM-1), and laser-plasma source coupling optics (CHARMS) using custom and non-sequential ray tracing}
        \resumeItem{Deliver results tailored to mission needs for technical trades and formal reviews}
      \resumeItemListEnd

    \resumeSubheading
      {Optics Lead for HARMONI Spectrograph System}{July 2016 -- April 2021}
      {University of Oxford}{Oxford, UK}
      \resumeItemListStart
        \resumeItem{Led optical design, modeling, and analysis of visible \& near-IR optics for key systems of science instrument for 39-meter ELT}
        \resumeItem{Developed Python library to simulate alignment and cryogenic testing of as-built optics}
        \resumeItem{Modeled (RCWA), procured, and tested VPHG; collaborated with KOSI and Wasatch; visited INAF and Syzygy VPHG labs}
        \resumeItem{Designed and built spectroradiometer for VPHG efficiency qualification; preliminary measurements presented at SPIE}
        \resumeItem{Led design and procurement of prototype off-axis mirror; coordinated with Glyndwr and tested alignment concept in lab}
        \resumeItem{Designed test setups, authored procedures, and performed laboratory measurements of prototype optics}
        \resumeItem{Assembled and aligned optical systems in ISO 7 cleanroom environment}
        \resumeItem{Collaborated across international partners on interface development and requirements flowdown}
        \resumeItem{Adapted to evolving project requirements in a cross-functional team using agile (Scrum) methods}
        \resumeItem{Supervised Ph.D. research and demonstrated optical fundamentals in undergraduate lab}
      \resumeItemListEnd

    \resumeSubheading
      {Graduate Researcher for Near-IR Spectrometer}{June 2011 -- June 2016}
      {University of Maryland / NASA GSFC}{Greenbelt, MD, USA}
      \resumeItemListStart
        \resumeItem{Ph.D. research on design, modeling, and cryogenic testing of the Rapid Infrared Imager-Spectrometer (RIMAS)}
        \resumeItem{Explored photonic OH-suppression architecture using fiber Bragg gratings for ground-based, near-IR spectroscopy}
        \resumeItem{Modeled (RCWA) and collaborated with Wasatch and LLNL to procure VPHGs and ZnSe grating prisms (grisms)}
        \resumeItem{Refined system-level requirements and formulated derived specifications}
        \resumeItem{Developed optical prescriptions, tolerancing models, and alignment strategies for compact refractive systems}
        \resumeItem{Authored manufacturing specifications and procured custom optics}
        \resumeItem{Assembled, aligned, and tested refractive imaging systems}
      \resumeItemListEnd
\resumeSubHeadingListEnd

%-----------EDUCATION-----------
\section{Education}
\resumeSubHeadingListStart
    \resumeSubheading
      {University of Maryland}{College Park, MD, USA}
      {Doctor of Philosophy in Astronomy}{August 2010 -- June 2016}
    \resumeSubheading
      {George Washington University}{Washington, DC, USA}
      {Bachelor of Science in Physics, Minor in Computer Science}{August 2006 -- May 2010}
\resumeSubHeadingListEnd

%-----------SERVICE & LEADERSHIP-----------
\section{Service \& Leadership}
\begin{itemize}[leftmargin=0.15in, label={}, noitemsep, topsep=0pt]
    \small{
        \item{\textbf{Subject Matter Expert}{ in optical components for NASA source evaluation and SBIR proposal review (2021-present)}}
        \item{\textbf{Peer and milestone reviewer}{ for NASA missions (2020-present)}}
        \item{\textbf{Mentor}{ to interns and early-career colleagues in optics and scientific programming (2016-present)}}
        \item{\textbf{Reviewer}{ for JATIS manuscripts on spaceborne UV instrumentation (2025)}}
        \item{\textbf{Guest editor}{ for MDPI Aerospace special issue, "Space Telescopes \& Payloads" (2023)}}
        \item{\textbf{Interviewer}{ at Ph.D. student milestones for the University of Oxford, Physics Department (2017–2021)}}
    }
\end{itemize}

%-----------SELECTED PRESENTATIONS-----------
\section{Selected Presentations}
\begin{itemize}[leftmargin=0.15in, label={}, noitemsep, topsep=0pt]
    \small{
        \item{\textbf{Paris Observatory}{: Invited seminar for international audience on optical modeling and simulation using ZOS-API (2021)}}
        \item{\textbf{NASA GSFC ISTD}{: Invited presentation for Division leadership on optical analysis workflow automation for CCRS (2021)}}
        \item{\textbf{Zemax Envision Europe 2020}{: Invited talk on TMA alignment simulations using ZOS-API (2020)}}
        \item{\textbf{MIT Kavli Institute}{: Invited seminar on HARMONI spectrographs (2019)}}
    }
\end{itemize}

%-----------SELECTED PUBLICATIONS-----------
\section{Selected Publications}
\begin{itemize}[leftmargin=0.15in, label={}, noitemsep, topsep=0pt]
    \small{
        \item{Bos, B.J., et al. (including \textbf{Capone, J.I.}) (2024). Vision System for the Mars Sample Return Capture Containment and Return System (CCRS). \textit{Aerospace}, 11(6), 456. \href{https://doi.org/10.3390/aerospace11060456}{doi:10.3390/aerospace11060456}}
        \item{\textbf{Capone, J.}, et al. (2020). HARMONI: first light spectroscopy for the ELT: simulating the alignment of a three-mirror anastigmat. \textit{Proc. SPIE}, 11447, 114472N. \href{https://doi.org/10.1117/12.2561519}{doi:10.1117/12.2561519}}
        \item{\textbf{Capone, J.I.}, et al. (2018). Near infrared throughput and stray light measurements of diffraction gratings for ELT-HARMONI. \textit{Proc. SPIE}, 10706, 1070644. \href{https://doi.org/10.1117/12.2310086}{doi:10.1117/12.2310086}}
        \item{\textbf{Capone, J.I.}, et al. (2014). Cryogenic optical systems for the rapid infrared imager/spectrometer (RIMAS). \textit{Proc. SPIE}, 9147, 914736. \href{https://doi.org/10.1117/12.2055503}{doi:10.1117/12.2055503}}
        \item{\textit{Full publication list available at:} \href{https://jicapone.github.io/CV/pubs/}{jicapone.github.io/CV/pubs/}}
    }
\end{itemize}

\end{document}