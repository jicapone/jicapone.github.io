\documentclass[letterpaper,11pt]{article}

%----------PACKAGES----------
\usepackage{latexsym}
\usepackage[empty]{fullpage}
\usepackage{titlesec}
\usepackage{marvosym}
\usepackage[usenames,dvipsnames]{color}
\usepackage{verbatim}
\usepackage{enumitem}
\usepackage[hidelinks]{hyperref}
\usepackage{fancyhdr}
\usepackage[english]{babel}
\usepackage{tabularx}
\usepackage{stix}
\usepackage{changepage}
\input{glyphtounicode}


%----------FONT OPTIONS----------
% Sans-serif options:
% \usepackage[sfdefault]{FiraSans}
% \usepackage[sfdefault]{roboto}
% \usepackage[sfdefault]{noto-sans}
% \usepackage[default]{sourcesanspro}
% 
% Serif options:
% \usepackage{CormorantGaramond}
% \usepackage{charter}

%----------PAGE SETUP----------
\pagestyle{fancy}
\fancyhf{} % clear all header and footer fields
\fancyfoot{}
\renewcommand{\headrulewidth}{0pt}
\renewcommand{\footrulewidth}{0pt}

% Adjust margins
\addtolength{\oddsidemargin}{-0.5in}
\addtolength{\evensidemargin}{-0.5in}
\addtolength{\textwidth}{1in}
\addtolength{\topmargin}{-.5in}
\addtolength{\textheight}{1.0in}

\urlstyle{same}
\raggedbottom
\raggedright
\setlength{\tabcolsep}{0in}

% Sections formatting
\titleformat{\section}{
  \vspace{-4pt}\raggedright\large
}{}{0em}{}[\color{black}\titlerule \vspace{-5pt}]

% Ensure that generated PDF is machine readable/ATS parsable
\pdfgentounicode=1

%----------CUSTOM COMMANDS----------
\newcommand{\resumeItem}[1]{
  \item\small{
    {#1 \vspace{-2pt}}
  }
}

\newcommand{\resumeSubheading}[4]{
  \vspace{-2pt}\item
    \begin{tabular*}{0.97\textwidth}[t]{l@{\extracolsep{\fill}}r}
      \textbf{#1} & #2 \\
      \textit{\small#3} & \textit{\small #4} \\
    \end{tabular*}\vspace{-7pt}
}

\newcommand{\resumeSubSubheading}[2]{
    \item
    \begin{tabular*}{0.97\textwidth}{l@{\extracolsep{\fill}}r}
      \textit{\small#1} & \textit{\small #2} \\
    \end{tabular*}\vspace{-7pt}
}

\newcommand{\resumeProjectHeading}[2]{
    \item
    \begin{tabular*}{0.97\textwidth}{l@{\extracolsep{\fill}}r}
      \small#1 & #2 \\
    \end{tabular*}\vspace{-7pt}
}

\newcommand{\resumeSubItem}[1]{\resumeItem{#1}\vspace{-4pt}}

%\renewcommand\labelitemii{$\vcenter{\hbox{\tiny$\bullet$}}$}
%\renewcommand\labelitemiii{$\vcenter{\hbox{\tiny$\bullet$}}$}
\renewcommand\labelitemii{\textbullet}
\renewcommand\labelitemiii{\textbullet}

\newcommand{\resumeSubHeadingListStart}{\begin{itemize}[leftmargin=0.15in, label={}]}
\newcommand{\resumeSubHeadingListEnd}{\end{itemize}}
\newcommand{\resumeItemListStart}{\begin{itemize}[leftmargin=0.15in]}
\newcommand{\resumeItemListEnd}{\end{itemize}\vspace{-5pt}}

%----------DOCUMENT BEGINS HERE----------
\begin{document}

\begin{center}
    \textbf{\Huge \scshape John I. Capone, Ph.D.} \\ \vspace{1pt}
    \small Ellicott City, Maryland, USA \\ \vspace{1pt}
    \small jicapone.optics@gmail.com \\ \vspace{1pt}
    \small  609-915-6213\\ \vspace{2pt}
\end{center}

\begin{adjustwidth}{50pt}{50pt}
\raggedright
\small Optical scientist and engineer with 14+ years of experience advancing scientific and spaceflight instrumentation from concept to flight. Skilled in high-fidelity modeling and hands-on implementation, including prototyping, lab measurements, automated test setups, and optical alignment. Developer of open-source Python libraries for ray tracing and wavefront analysis, exceeding the capabilities of commercial software. Proven leader and cross-functional collaborator, delivering innovative, system-aware solutions that bridge science objectives and engineering constraints.
\end{adjustwidth}

%-----------SKILLS-----------
\section{Skills}
\begin{itemize}[leftmargin=0.15in, label={}]
    \small{
        \item\begin{minipage}[t]{\linewidth}
            \textbf{Optical Design}: \hangindent=25pt \hangafter=1 \raggedright System architecture; Compact imaging systems; Holographic and diffractive optics; Freeform surfaces; Illumination design; Requirements flowdown and validation; Verification planning; Performance budgets
        \end{minipage}
        \item\begin{minipage}[t]{\linewidth}
            \textbf{Optical Modeling}: \hangindent=25pt \hangafter=1 \raggedright High-fidelity ray tracing; Wavefront propagation; Extended Nijboer-Zernike (ENZ) analysis; Rigorous Coupled-Wave Analysis (RCWA); Radiometry; Interferometry; Finite-Difference Time-Domain (FDTD); Stray light; End-to-end modeling; Tolerancing; Monte Carlo (MC) analysis
        \end{minipage}
        \item\begin{minipage}[t]{\linewidth}
            \textbf{Hardware Implementation}: \hangindent=25pt \hangafter=1 \raggedright Custom metrology design; Data analysis; Specifications; Drawings; Vendor technical interface; Prototyping; Alignment, integration, \& test (AIT); Fizeau interferometry, LUPI; Spectroradiometry; CMMs (coordinate measuring machines); Cryogenic and vacuum testing; Cleanroom operations; Experimental nuclear physics
        \end{minipage}
        \item\begin{minipage}[t]{\linewidth}
            \textbf{Software Development \& Tools}: \hangindent=25pt \hangafter=1 \raggedright Python; PyTorch; Git; Custom ray tracing libraries; Zemax OpticStudio (ZOS) API; GSolver; Code V; FRED; MATLAB; LabVIEW; SolidWorks; Autodesk Inventor; VS Code; C
        \end{minipage}
    }
\end{itemize}


%-----------EXPERIENCE-----------
\section{Experience}
\resumeSubHeadingListStart
    \resumeSubheading
      {Optical Design and Analysis Group Lead}{November 2024 -- Present}
      {NASA Goddard Space Flight Center}{Greenbelt, MD, USA}
      \resumeItemListStart
        \resumeItem{Coordinated with Optics Branch leadership (100+ personnel) to align group capabilities with project needs, facilitating staffing discussions and technical planning for a team of 16 optical engineers and scientists, including 10 Ph.D.-level staff}
        \resumeItem{Organized and delivered technical talks to promote knowledge transfer across projects}
        \resumeItem{Coordinated peer reviews and updated group work instruction to strengthen engineering rigor, foster collaboration, and improve quality control}
      \resumeItemListEnd

    \resumeSubheading
      {Optics Lead for NASA Payloads}{April 2021 -- Present}
      {NASA Goddard Space Flight Center}{Greenbelt, MD, USA}
      \resumeItemListStart
        \resumeItem{Reduced CCRS Vision System from 13 to 7 cameras two months after onboarding using custom system-level analysis pipeline}
        \resumeItem{Added color imaging to LExSO Hi Res Imager with minimal impact to cost, schedule, or complexity}
        \resumeItem{Saved CUVIS cost and schedule by identifying existing Optical Ground Support Equipment (OGSE)}
        \resumeItem{Applied COTS components across missions to reduce cost, mitigate risk, and accelerate schedules}
        \resumeItem{Designed and optimized illumination systems using custom and non-sequential ray tracing for CCRS, OSAM-1, and CHARMS}
        \resumeItem{Developed advanced Python tools (GPU ray tracing, ENZ PSF, heterodyne DWS) exceeding commercial software capabilities}
        \resumeItem{Validated CCRS vision requirements using synthetic image data and machine learning to assess compliance}
        \resumeItem{Led IRAD research into 3D printed optical prototyping, achieving 3× improvement in surface form accuracy}
      \resumeItemListEnd

    % did I do anything to save cost or schedule?
    \resumeSubheading 
      {Optics Lead for HARMONI Spectrograph System}{July 2016 -- April 2021}
      {University of Oxford}{Oxford, UK}
      \resumeItemListStart
        \resumeItem{Led optical design and analysis of €4M Spectrograph system for first-light integral field spectrograph on the 39-meter ELT}
        \resumeItem{Designed refractive visible and near-IR cameras operated at 130 K; developed solutions with limited lens materials due to size}
        \resumeItem{Delivered realistic as-built MC analysis with custom Python library + ZOS API to simulate manufactured optics, including mid-spatial errors, and AIT procedures}
        \resumeItem{Modeled, procured, and tested custom VPH gratings; developed spectroradiometer; led technical interactions with KOSI, Wasatch, INAF, and Syzygy}
        \resumeItem{Led prototype mirror procurement and vendor engagement; authored RFQs, evaluated responses, and conducted site visits}
        \resumeItem{Directed stray light modeling and baffle design; supported internal and ESO reviews with documentation and presentations}
        \resumeItem{Supervised Ph.D. research and demonstrated optical fundamentals in undergraduate lab}
      \resumeItemListEnd

    \resumeSubheading
      {Graduate Researcher – Rapid Infrared Imager-Spectrometer (RIMAS)}{June 2011 -- June 2016}
      {University of Maryland / NASA GSFC}{Greenbelt, MD, USA}
      \resumeItemListStart
        \resumeItem{Led cryogenic optical design and modeling of RIMAS, a compact near-IR imaging spectrometer for rapid transient follow-up on the 4.3-m DCT}
        \resumeItem{Refined system-level requirements, developed error budgets, and derived testable specifications}
        \resumeItem{Developed Zemax models, tolerancing analyses, and alignment strategies for refractive optics operating at 60 K}
        \resumeItem{Modeled, procured, and tested custom VPH gratings and ZnSe grisms; coordinated with Wasatch and LLNL to meet specifications}
        \resumeItem{Assembled and aligned optical systems in ambient conditions; verified performance in cryogenic vacuum chamber}
        \resumeItem{Explored photonic OH-suppression using fiber Bragg gratings to improve ground-based IR sensitivity}
        \resumeItem{Authored manufacturing specifications, produced mechanical drawings, and interfaced with vendors to deliver custom optics}
      \resumeItemListEnd

\resumeSubHeadingListEnd

%-----------EDUCATION-----------
\section{Education}
\resumeSubHeadingListStart
    \resumeSubheading
      {University of Maryland}{College Park, MD, USA}
      {Doctor of Philosophy in Astronomy}{August 2010 -- June 2016}
    \resumeSubheading
      {George Washington University}{Washington, DC, USA}
      {Bachelor of Science in Physics, Minor in Computer Science}{August 2006 -- May 2010}
\resumeSubHeadingListEnd

%-----------SERVICE & LEADERSHIP-----------
\section{Service \& Leadership}
\begin{itemize}[leftmargin=0.15in, label={}, noitemsep, topsep=0pt]
    \small{
        \item{\textbf{Subject Matter Expert}{ in optical components for NASA source evaluation and SBIR proposal review (2021-present)}}
        \item{\textbf{Peer and milestone reviewer}{ for NASA missions (2020-present)}}
        \item{\textbf{Mentor}{ to interns and early-career colleagues in optics and scientific programming (2016-present)}}
        \item{\textbf{Reviewer}{ for JATIS manuscripts on spaceborne UV instrumentation (2025)}}
        \item{\textbf{Guest editor}{ for MDPI Aerospace special issue, "Space Telescopes \& Payloads" (2023)}}
        \item{\textbf{Reviewer}{ at Ph.D. student milestones for the University of Oxford, Physics Department (2017–2021)}}
    }
\end{itemize}

%-----------SELECTED PRESENTATIONS-----------
\section{Selected Presentations}
\begin{itemize}[leftmargin=0.15in, label={}, noitemsep, topsep=0pt]
    \small{
        \item{\textbf{Paris Observatory}{: Invited seminar for international audience on optical modeling and simulation using ZOS-API (2021)}}
        \item{\textbf{NASA GSFC ISTD}{: Invited presentation for Division leadership on optical analysis workflow automation for CCRS (2021)}}
        \item{\textbf{Zemax Envision Europe 2020}{: Invited talk on TMA alignment simulations using ZOS-API (2020)}}
        \item{\textbf{MIT Kavli Institute}{: Invited seminar on HARMONI spectrographs (2019)}}
    }
\end{itemize}

%-----------SELECTED PUBLICATIONS-----------
\section{Selected Publications}
\begin{itemize}[leftmargin=0.15in, label={}, noitemsep, topsep=0pt]
    \small{
        \item{Bos, B.J., et al. (including \textbf{Capone, J.I.}) (2024). Vision System for the Mars Sample Return Capture Containment and Return System (CCRS). \textit{Aerospace}, 11(6), 456.}
          \item{\textbf{Capone, J.}, et al. (2020). HARMONI: first light spectroscopy for the ELT: simulating the alignment of a three-mirror anastigmat. \textit{Proc. SPIE}, 11447, 114472N.}
          \item{\textbf{Capone, J.I.}, et al. (2018). Near infrared throughput and stray light measurements of diffraction gratings for ELT-HARMONI. \textit{Proc. SPIE}, 10706, 1070644.}
          \item{\textbf{Capone, J.I.}, et al. (2014). Cryogenic optical systems for the rapid infrared imager/spectrometer (RIMAS). \textit{Proc. SPIE}, 9147, 914736.}
          \bigskip
          \item{Full publication list available at: \textit{jicapone.github.io/CV/pubs/}}
    }
\end{itemize}

\end{document}